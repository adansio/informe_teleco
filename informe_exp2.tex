\documentclass[12pt]{article}
\usepackage[T1]{fontenc}
\usepackage[utf8]{inputenc}
\usepackage[spanish]{babel}
\usepackage{graphicx}
\usepackage{float}
\usepackage[left=2.7cm,top=2.5cm,right=2.7cm,bottom=2.5cm]{geometry}
\usepackage{url}
\usepackage{setspace}
\usepackage{fancyhdr}

\hyphenation{co-rres-pon-de}
% encabezados
\lhead[]{\footnotesize Universidad Técnica Federico Santa María. \rightmark}
\chead[]{}
\rhead[\footnotesize \leftmark]{}
\renewcommand{\headrulewidth}{0.5pt}

%\pagestyle{fancy}

\title{\large Informe final Experiencia \#2\\ \huge Medición de Canal Inalámbrico Fijo en 
Banda Angosta\\}
\author{\\ \ \\ \ \\ \ \\ \ \\ \ \\ \ \\ \ \\ \ \\ \ \\
\small
\begin{tabular}{ l l l }
Profesor & Luis Vinnett &\\
& & \\
Asignatura & Laboratorio de Telecomunicaciones \\
& & \\
Sigla & ELO-352 & \\
& & \\
Integrantes & Eduardo Barra & 2730038-3\\
&	Adan Morales & 2830042-5 \\
&	Ricardo Salinas & 2830015-8\\
& & \\
Grupo & 2 &\\
\Large	& & \\
& & \\
& & \\
& & \\
& & \\
\normalsize	& & \\
\end{tabular} }
\date{\today}

\begin{document}
\maketitle
\thispagestyle{fancy}

\newpage
\thispagestyle{fancy}
\tableofcontents

\newpage
\pagestyle{fancy}
\section{Introducción}

\begin{spacing}{1.5}
Las comunicaciones inalámbricas son más complejas de lo que parecen, ya que las señales estan 
sometidas constantemente a desvanecimientos, ya sean especiales y/o temporales, debido a 
movimiento del Tx, Rx u objetos entre ambos, provocando estos caminos, multitrayectorias variables, 
lo que afecta que cambie la naturaleza del canal inalámbrico haciendo necesario evaluar parámetros 
en forma estadística, como es el tiempo de coherencia y el ancho de banda de coherencia.\\

Luego de la experiencia de laboratorio realizada la cual consiste en realizar mediciones del 
canal inalámbrico en distintos escenarios, ya sea enlace LOS, NLOS (situación con gente en transito 
entre Tx y Rx, influyendo en los multitrayectos), además, también se observará cómo afecta la 
orientación de la antena receptora respecto de la antena transmisora, ya sean estando paralela 
o perpendicular una de la otra.\\

Las mediciones se realizarán utilizando un “Virtual Instrument” de LabView, para luego realizar un 
análisis más estadístico, y determinar cuales son los mejores escenarios para transmitir a través 
de un canal inalámbrico, a su vez de observar y concluir por qué al cambiar de escenario la señal 
podría llegar con bastantes pérdidas y muy degradada.\\
\end{spacing}

\renewcommand{\baselinestretch}{1}
\newpage
\section{Procedimiento}

Se comienza la experiencia ubicando una única antena transmisora de dipolo en un punto fijo a 
unos 2[m] del piso, la cual emitirá una señal a 2,4 [GHz], luego la antena receptora se ubicará 
a unos 34 a 38 [m] de la antena anterior, para luego proceder con el Instrumento Virtual Frankonia 
Power Meter, realizando capturas cada 20 [ms] entre unos 10 y 15 [min] aproximadamente.\\

Las condiciones en las cuales se realizaron las mediciones son las siguientes:
\begin{center}
	\begin{tabular}{| c | c | c | c | c | c |} \hline
   & LOS/NLOS & Polarización & Condición & Duración & Distancia  \\ 
   & 		& en Rx & ambiental & [min] & [m]  \\ \hline 
Medición 1 & LOS	& Vertical	& Sin gente	& 15 & 35 \\ \hline
Medición 2 & LOS	& Horizontal	& Sin gente	& 15 & 35 \\ \hline
Medición 3 & NLOS	& Horizontal	& Sin gente	& 12 & 38	\\ \hline
Medición 4 & NLOS	& Horizontal	& Con gente	& 10 & 38	\\ \hline
Medición 5 & NLOS	& Vertical	& Con gente	& 10 & 38	\\ \hline
Medición 6 & NLOS	& Vertical	& Sin gente	& 15 & 38	\\ \hline
Medición 23 & LOS	& Vertical	& Con gente	& 2 (fallida) & 35	\\ \hline
Medición 24 & LOS	& Vertical	& Con gente	& 10 & 35	\\ \hline
Medición 25 & LOS	& Horizontal	& Con gente	& 10 & 35	\\  \hline
Medición 26 & LOS	& Horizontal	& Con gente	& 10 &	34 (En altura)\\ \hline
Medición 27 & LOS	& Vertical	& Sin gente	& 10 & 34 (En altura)	\\ \hline
	\end{tabular}
\end{center}


Además se realizaron mediciones para observar como varía la señal cuando se mueve el Rx 
unos 15 [cm], esto es cercano a la longitud de onda 0,125 [m]. Esta medición fué bajo las 
siguientes condiciones:\\
-LOS\\ -Polarización vertical\\ -Sin gente\\ -Duración de 5 [min]\\ -A 35 [m] del Tx.\\

Acontinuación un esquema desde una vista superior.\\
\begin{center}
	\begin{tabular}{| c | c | c | c |} \hline
	Medición 7 & Medición 8 & Medición 9 & Medición 10 \\ \hline
	Medición 14 & Medición 13 & Medición 12 & Medición 11 \\ \hline
	Medicion 15 & Medición 16 & Medición 17 & Medición 18 \\ \hline
	Medición 22 & Medición 21 & Medición 20 & Medición 19 \\ \hline
	\end{tabular}
\end{center}
\footnotesize 
\begin{center}
Tabla 2: Diagrama de ubicación de los puntos de recepción, visto desde arriba.
\end{center}
\normalsize

\newpage
\section{Análisis}

\paragraph{1.- Para las mediciones de fading temporal, determine el factor K y tiempo de coherencia 
de las mediciones. Para ello, analice el record completo y luego divídalo en 3 segmentos de 
5 minutos. Ello permitirá analizar la consistencia de los resultados. ¿Qué fenómeno se observa 
al obtener el factor K de los segmentos de 5 minutos en comparación de aquél correspondiente al 
del record completo? ¿Qué se puede concluir del tiempo de coherencia? Separe los resultados en 
base a tráfico observado durante la medición (horario de recreo entre clases!) y grado de 
obstrucción. Con respecto a esto último es habitual especificar el “excess path loss” que es 
la pérdida por sobre lo que predice la ecuación de Friis para un enlace de igual longitud.\\}
\small
\begin{center}
	\begin{tabular}{| c | c | c | c | c | c |} \hline
Med & Descripción & K-total & K 1/3 & K 2/3 & K 3/3 \\ \hline
1	& LOS		&	1.8123  &  1.9674 	&  2.0048  	&  1.5455 \\
	& Vertical	&			&			&			&			\\
	& Sin gente &			&			&			&			\\ \hline
2	& LOS 		&	1.2846  &  2.0454   &  2.2164   & 18.8217	\\
	& Horizontal&			&			&			&			\\
	& Sin gente &			&			&			&			\\ \hline
3	& NLOS		& 104.6428  & 107.6862  & 153.1641  & 131.9275	\\
	& Horizontal&			&			&			&			\\
	& Sin gente	&			&			&			&			\\ \hline
4	& NLOS		& 79.0962   & 87.1520   &  76.0635  & 80.3491	\\ 
	& Horizontal&			&			&			&			\\
	& Con gente	&			&			&			&			\\ \hline
5	& NLOS		& 135.5027  &119.1709  	& 164.7176  & 196.9461	\\
	& Vertical	&			&			&			&			\\
	& Con gente	&			&			&			&			\\ \hline
6	& NLOS		& 225.9990  & 194.3546  & 262.9200  & 246.4110	\\
	& Vertical	&			&			&			&			\\ 
	& Sin gente	&			&			&			&			\\ \hline
24	& LOS		&  1.6423   & 1.8772    & 1.6558    & 1.6288	\\ 
	& Vertical	&			&			&			&			\\
	& Con gente	&			&			&			&			\\ \hline
25	& LOS		& 17.3114   & 18.1255   & 40.2994   & 17.0116	\\ 
	& Horizontal&			&			&			&			\\
	& Con gente	&			&			&			&			\\ \hline
26	& LOS 		& 44.6654   & 51.7480   & 100.9218  & 112.0481	\\ 
	& Horizontal&			&			&			&			\\
	& Con gente	&			&			&			&			\\ \hline
27	& LOS		& 1.4003    & 1.8735    & 1.8051    & 12.6134	\\
	& Vertical	&			&			&			&			\\
	& Sin gente	&			&			&			&			\\ \hline
	\end{tabular}
\end{center}
\normalsize

Primero, utilizando matlab se obtiene un $K_{total}$ de muestreo, y un $K$ para cada tramo
de la medición, siendo la medición total dividida en 3 partes de la misma duración. 

Realizando una observación sobre los resultados, en base al factor K total obtenido
de forma experimental, se obtiene una mejor comunicación o estado del canal inalámbrico 
cuando el enlace es en línea vista, luego, con gente en la zona de Fresnel comienza a 
verse un poco afectada la comunicación, de todas formas dependiendo de la polarización
de la antena. Para el caso de NLOS se observa que la comunicación mejora levemente al haber
gente en la zona de Fresnel, esto puede deberse a que aumentan más los multitrayectos
hacia la antena receptora, y además la señal mejora levemente si la polarización de la 
antena es horizontal, siendo algo extraño ya que la comunicación deberia ser mejor si 
la polarización es la misma para ambas antenas.

Para la medición 26 y 27, se observa mejor el caso en donde la comuniciación según el 
parámetro K es mucho mejor en LOS, polarización vertical y sin gente, en vez de 
LOS, polarización horizontal y con gente.\\

Para el tiempo de coherencia se establece como el tiempo durante el cual la autocovarianza 
de la señal está por sobre 0,5, para proceder con esto se programa la función en matlab 
obteniendo los siguientes resultados (incluyendo las mediciónes en altura).
\begin{center}
	\begin{tabular}{| c | c | c | c | c |} \hline
Medición & $T_{coherencia-total}$ & $T_{coherencia1/3}$ & $T_{coherencia2/3}$ & 
$T_{coherencia3/3}$\\ \hline
1	& 0.2869 &	0.3368 &   0.2807 &  1.5719 \\ \hline
2	& 74.9660 &	1.6498 &   3.6135 & 25.5695 \\ \hline
3	& 0.9613 &	1.0815 &   1.3864 &  0.3882 \\ \hline
4	& 0.1819 &	0.2122 &   0.2728 & 0.2424 \\ \hline
5	& 0.5899 &	0.7147 &   0.7486 &  0.4764 \\ \hline
6	& 0.1701 &	0.1767 &   0.3337 &  0.3729 \\ \hline
24	& 0.7722 & 2.0687  &  0.6618  &  0.6204	\\ \hline
25	& 2.6154 & 1.1077  &  1.4305  &  3.3687\\ \hline
26	& 21.8965 & 7.7498  &  3.4293  & 18.5625 \\ \hline
27	& 49.9193 & 1.4206 &  19.6023 &   2.6512	\\ \hline
	\end{tabular}
\end{center}

El código utilizado fué el que se muestra a continuación.

\small
\begin{verbatim}
function [output] = Coherencia (A)
B = load (A);
datos = B(:,1); 
		    
%se segmenta los datos en 3 partes
primeraMitadDatos = datos(1:floor(length(datos)/3));
segundaMitadDatos = datos(floor(length(datos)/3):floor(2*length(datos)/3));
terceraMitadDatos = datos(floor(2*length(datos)/3):length(datos)); 
							    
%cantidad de mediciones totales realizadas
DatosTotales = length(datos);
LprimeraMitad = length(primeraMitadDatos);
LsegundaMitad = length(segundaMitadDatos);
LterceraMitad = length(terceraMitadDatos);
								    
%se obtiene la autocovarianza de los datos
datosAutoCorrTotal = (autocorr(datos,DatosTotales-1));
datosAutoCorrPrimera = (autocorr(primeraMitadDatos,LprimeraMitad-1));
datosAutoCorrSegunda = (autocorr(segundaMitadDatos,LsegundaMitad-1));
datosAutoCorrTercera = (autocorr(terceraMitadDatos,LterceraMitad-1));
																		    
%a partir de la autocorrelacion obtenemos el tiempo de coherencia para
%cada segmento buscando los valores de autocorrelacion mayor a 0,5
TcTotal = find(datosAutoCorrTotal<0.5,1);
TcTotal = 15*60*TcTotal/length(datos);
																							     
TcPrimera = find(datosAutoCorrPrimera<0.5,1);
TcPrimera = 5*60*TcPrimera/length(primeraMitadDatos);
																				      
TcSegunda = find(datosAutoCorrSegunda<0.5,1);
TcSegunda = 5*60*TcSegunda/length(segundaMitadDatos);
																								      
TcTercera = find(datosAutoCorrTercera<0.5,1);
TcTercera = 5*60*TcTercera/length(terceraMitadDatos);
																							      
%coordenada x
x = 0:15/(length(datos)-1):15;
																		      
%se grafica autocovarianza en funcion del numero de datos 
plot (x,datosAutoCorrTotal);
xlabel('minutos');
ylabel('autocovarianza');
output = [TcTotal TcPrimera TcSegunda TcTercera];
end
\end{verbatim}
\normalsize

Como se mencionó anteriormente el tiempo de coherencia se refiere al tiempo en el cual 
el canal no varía, por lo que es de esperar que para las mediciones efectuadas en 
escenarios sin gente el tiempo de coherencia fuera más largo, que los otros casos, puesto 
que existen menores objetos dispersores que pudieran alterar el canal de transmisión.
Según los datos calculados esto se refleja parcialmente ya que para las mediciones numero 
2 y 27 que son escenarios sin gente se refleja una alta correlación, sin embargo, para 
otras mediciones efectuadas sin gente no se aprecia un valor de correlación tan alto como 
los primeros.  Lo anterior puede ser causa a que los altos valores de correlación se 
apreciaron en escenarios de línea vista.\\
Respecto al Path-Loss se puede decir que este corresponde a la pérdida por sobre lo que 
predice la ecuación de Friis para un enlace de igual longitud (en nuestro caso 35[m] aprox.).\\

La ecuación de Friss para caracterizar el enlace nos permite obtener una relación
entre la potencia transmitida y recibida, en un escenario de línea vista (LOS). Por otro lado, 
la antena transmisora teóricamente, esta irradiando a una potencia de 10dBm.\\
\newpage
Con las distancias y la expresión $ -20 log \left ( \frac{\lambda}{4 \cdot \pi \cdot d} \right )$
se calculó el path loss obteniendose la siguiente tabla:

\begin{center}
	\begin{tabular}{| c | c | c | c |} \hline
   &   & Distancia [m] & PL [dB] \\ \hline 
Medición 1 & LOS	& 35	& 70,9	\\ \hline
Medición 2 & LOS	& 35	& 70,9	\\ \hline
Medición 3 & NLOS	& 38	& 71,6	\\ \hline
Medición 4 & NLOS	& 38	& 71,6	\\ \hline
Medición 5 & NLOS	& 38	& 71,6	\\ \hline
Medición 6 & NLOS	& 38	& 71,6	\\ \hline
Medición 24 & LOS	& 35	& 70,9	\\ \hline
Medición 25 & LOS	& 35	& 70,9	\\ \hline
Medición 26 & LOS	& 34	& 70,6	\\ \hline
Medición 26 & LOS	& 34	& 70,6	\\ \hline
	\end{tabular}
\end{center}

\paragraph{2.- Estime las autocovarianzas de los datos de desvanecimientos temporales usando los 
intervalos de 15 minutos. En cada gráfico indique además el valor del factor K. (como las 
autocovarianzas están autoescaladas a uno este dato es necesario para completar la información).\\}

Para la autocovarianza de las muestras en el caso de fading temporal, se obtienen usando la
siguiente función en matlab.

\small 	
\begin{verbatim}
% Autocovarianza
%N: Numero total de datos.
%(x): se va variando manualmente entre 1-6,24-27.
datos=load('Grupo2/med(x).txt'); 

%Obtencion de datos necesarios
for i=1:length(datos)
	new_datos(i)=datos(i,1);
end
N=length(new_datos); 

%Graficos de autocorrelacion
acorr=(autocorr(new_datos,N-1));

%Rango de tiempo.
x = 0:15/(length(new_datos)-1):15;
plot (x,acorr)
xlabel('Tiempo [min]');
ylabel('Autocovarianza [s/u]');
%(y) es modificado manualmente siendo el K correspondiente a (x).  
legend('Autocovarianza - K=(y)'); 

end
\end{verbatim}
\normalsize

Donde luego se obtienen los gráficos de a continuación.

\begin{figure}[H]
  \centering
        \includegraphics[width=0.65\textwidth]{med1}
		\caption{\footnotesize 
		Gráfico de autocovarianza de medición 1. LOS, Polarización Vertical, Sin gente.}
\label{fig:med1}
\end{figure}

\begin{figure}[H]
  \centering
        \includegraphics[width=0.65\textwidth]{med2}
		\caption{\footnotesize
		Gráfico de autocovarianza de medición 2. LOS, Polarización Horizontal, Sin gente.}
\label{fig:med2}
\end{figure}

\begin{figure}[H]
  \centering
        \includegraphics[width=0.65\textwidth]{med3}
		\caption{\footnotesize
		Gráfico de autocovarianza de medición 3. NLOS, Polarización Horizontal, Sin gente.}
\label{fig:med3}
\end{figure}

\begin{figure}[H]
  \centering
        \includegraphics[width=0.65\textwidth]{med4}
		\caption{\footnotesize
		Gráfico de autocovarianza de medición 4. NLOS, Polarización Horizontal, Con gente.}
\label{fig:med4}
\end{figure}

\begin{figure}[H]
  \centering
        \includegraphics[width=0.65\textwidth]{med5}
		\caption{\footnotesize
		Gráfico de autocovarianza de medición 5. NLOS, Polarización Vertical, Con gente.}
\label{fig:med5}
\end{figure}

\begin{figure}[H]
  \centering
        \includegraphics[width=0.65\textwidth]{med6}
		\caption{\footnotesize
		Gráfico de autocovarianza de medición 6. NLOS, Polarización Vertical, Sin gente.}
\label{fig:med6}
\end{figure}

\begin{figure}[H]
  \centering
        \includegraphics[width=0.65\textwidth]{med24}
		\caption{\footnotesize
		Gráfico de autocovarianza de medición 24. LOS, Polarización Vertical, Con gente.}
\label{fig:med24}
\end{figure}

\begin{figure}[H]
  \centering
        \includegraphics[width=0.65\textwidth]{med25}
		\caption{\footnotesize
		Gráfico de autocovarianza de medición 25. LOS, Polarización Horizontal, Con gente.}
\label{fig:med25}
\end{figure}

\begin{figure}[H]
  \centering
        \includegraphics[width=0.65\textwidth]{med26}
		\caption{\footnotesize
		Gráfico de autocovarianza de medición 26. LOS (en altura), Polarización Horizontal, 
		Con gente.}
\label{fig:med26}
\end{figure}

\begin{figure}[H]
  \centering
        \includegraphics[width=0.65\textwidth]{med6}
		\caption{\footnotesize
		Gráfico de autocovarianza de medición 27. LOS (en altura), Polarización Vertical, 
		Sin gente.}
\label{fig:med27}
\end{figure}

\normalsize

\paragraph{3.- Intente ajustar una curva exponencial a cada autocovarianza y determine la constante 
de tiempo de decaimiento del mejor ajuste para cada caso.\\}

Basado en el código de la pregunta anterior se tiene el conjunto de datos para proceder 
a obtener su correspondiente curva exponencial. El código completo a continuación.

\begin{verbatim}
function [ x, acorr ] = pregunta2( file )
%PREGUNTA2 Summary of this function goes here
%   Detailed explanation goes here
% Autocovarianza
%N: Numero total de datos.
%(x): se va variando manualmente entre 1-6,24-27.
datos=load(file); 
N = length(datos);
disp(N);

%Obtencion de datos necesarios
for i=1:length(datos)
	  new_datos(i)=datos(i,1);
	  end

	  %Graficos de autocorrelacion
	  acorr=(autocorr(new_datos,N-1));

	  %Rango de tiempo.
	  x = 0:15/(length(new_datos)-1):15;
	  plot (x,acorr)
	  xlabel('Tiempo [min]');
	  ylabel('Autocovarianza [s/u]');
	  %(y) es modificado manualmente siendo el K correspondiente a (x).  
	  legend('Autocovarianza - K=(y)'); 

	  end
\end{verbatim}

Luego usando la toolbox cftool se obtienen las curvas exponenciales que se muestran a 
a continuación, en las figuras \ref{fig:cmed1} a la \ref{fig:cmed27}.\\

En la textbox de cada gráfico se encuentran los parámetros de la curva, donde ``b'' 
corresponde a la constante de tiempo de cada medición.\\

\begin{figure}[H]
\hfill
\begin{minipage}[t]{.45\textwidth}
  \centering
        \includegraphics[width=1\textwidth]{curvas/med1}
		\caption{\footnotesize Curva exponencial sobre medición 1.}
\label{fig:cmed1}
\end{minipage}
\hfill
\begin{minipage}[t]{.45\textwidth}
  \centering
        \includegraphics[width=1\textwidth]{curvas/med2}
		\caption{\footnotesize Curva exponencial sobre medición 2.}
\label{fig:cmed2}
\end{minipage}
\end{figure}

\begin{figure}[H]
\hfill
\begin{minipage}[t]{.45\textwidth}
  \centering
        \includegraphics[width=1\textwidth]{curvas/med3}
		\caption{\footnotesize Curva exponencial sobre medición 3.}
\label{fig:cmed3}
\end{minipage}
\hfill
\begin{minipage}[t]{.45\textwidth}
  \centering
        \includegraphics[width=1\textwidth]{curvas/med4}
		\caption{\footnotesize Curva exponencial sobre medición 4.}
\label{fig:cmed4}
\end{minipage}
\end{figure}

\begin{figure}[H]
\hfill
\begin{minipage}[t]{.45\textwidth}
  \centering
        \includegraphics[width=1\textwidth]{curvas/med5}
		\caption{\footnotesize Curva exponencial sobre medición 5.}
\label{fig:cmed5}
\end{minipage}
\hfill
\begin{minipage}[t]{.45\textwidth}
  \centering
        \includegraphics[width=1\textwidth]{curvas/med6}
		\caption{\footnotesize Curva exponencial sobre medición 6.}
\label{fig:cmed6}
\end{minipage}
\end{figure}

\begin{figure}[H]
\hfill
\begin{minipage}[t]{.45\textwidth}
  \centering
        \includegraphics[width=1\textwidth]{curvas/med24}
		\caption{\footnotesize Curva exponencial sobre medición 24.}
\label{fig:cmed24}
\end{minipage}
\hfill
\begin{minipage}[t]{.45\textwidth}
  \centering
        \includegraphics[width=1\textwidth]{curvas/med25}
		\caption{\footnotesize Curva exponencial sobre medición 25.}
\label{fig:cmed25}
\end{minipage}
\end{figure}

\begin{figure}[H]
\hfill
\begin{minipage}[t]{.45\textwidth}
  \centering
        \includegraphics[width=1\textwidth]{curvas/med26}
		\caption{\footnotesize Curva exponencial sobre medición 26.}
\label{fig:cmed26}
\end{minipage}
\hfill
\begin{minipage}[t]{.45\textwidth}
  \centering
        \includegraphics[width=1\textwidth]{curvas/med27}
		\caption{\footnotesize Curva exponencial sobre medición 27.}
\label{fig:cmed27}
\end{minipage}
\end{figure}

En base a los gráficos obtenidos se puede decir que algunas mediciones presentaban demasiada 
distorsión en sus datos por lo que no se pudo modelar una curva con una exponencial normal 
si no con 2 exponenciales superpuestas a diferencia de otras, que son más exactas y pudieron 
ser modeladas con una exponencial simple.\\

Como comentario a partir de lo observado se agrega que la señal se distorsiona fácilmente 
arrojando datos erroneos durante una comunicación wireless.

\paragraph{4.- Para un mismo escenario, construya histogramas con los datos de fading temporal y 
con los de fading espacial y compare.\\ \ \\}

Comparación en escenario LOS, polarización vertical en figura \ref{fig:histo1} y \ref{fig:histo24}.

\begin{figure}[H]
\hfill
\begin{minipage}[t]{.45\textwidth}
  \centering
        \includegraphics[width=1\textwidth]{histo1}
		\caption{\footnotesize LOS, Polarización vertical, sin gente.}
\label{fig:histo1}
\end{minipage}
\hfill
\begin{minipage}[t]{.45\textwidth}
  \centering
        \includegraphics[width=1\textwidth]{histo24}
		\caption{\footnotesize LOS, Polarización vertical, con gente.}
\label{fig:histo24}
\end{minipage}
\end{figure}
\footnotesize
Se observa que para el caso sin gente, hay menores multitrayectos, por lo tanto menor cantidad 
de elementos interfiriendo el canal, dispersandose menos la señal.\\
\newpage
\normalsize
Comparación en escenario LOS, polarización horizontal en figura \ref{fig:histo2} y \ref{fig:histo25}.
\begin{figure}[H]
\hfill
\begin{minipage}[t]{.45\textwidth}
  \centering
        \includegraphics[width=1\textwidth]{histo2}
		\caption{\footnotesize LOS, Polarización horizontal, sin gente.}
\label{fig:histo2}
\end{minipage}
\hfill
\begin{minipage}[t]{.45\textwidth}
  \centering
        \includegraphics[width=1\textwidth]{histo25}
		\caption{\footnotesize LOS, Polarización horizontal, con gente.}
\label{fig:histo25}
\end{minipage}
\end{figure}
\footnotesize
Para la figura \ref{fig:histo2}, existe una alta concentración a 0 dB, pero con perdidas debido a la 
polarización, sin embargo, en el caso con gente, esta se concentra principalmente en torno
a los -7 dB, debido a la dispersión de la señal por la gente transitando.\\

\normalsize
Comparación en escenario NLOS, polarización vertical en figura \ref{fig:histo6} y \ref{fig:histo5}.
\begin{figure}[H]
\hfill
\begin{minipage}[t]{.45\textwidth}
  \centering
        \includegraphics[width=1\textwidth]{histo6}
		\caption{\footnotesize NLOS, Polarización vertical, sin gente.}
\label{fig:histo6}
\end{minipage}
\hfill
\begin{minipage}[t]{.45\textwidth}
  \centering
        \includegraphics[width=1\textwidth]{histo5}
		\caption{\footnotesize NLOS, Polarización vertical, con gente.}
\label{fig:histo5}
\end{minipage}
\end{figure}
\footnotesize
La diferencia no es realmente notoria para los casos anteriores, pero se puede decir, que para el 
caso de la figura \ref{fig:histo5} mejora levemente por efecto de los multitrayectos.\\
\newpage
\normalsize
Comparación en escenario NLOS, polarización horizontal en figura \ref{fig:histo3} y \ref{fig:histo4}.
\begin{figure}[H]
\hfill
\begin{minipage}[t]{.45\textwidth}
  \centering
        \includegraphics[width=1\textwidth]{histo3}
		\caption{\footnotesize NLOS, Polarización horizontal, sin gente.}
\label{fig:histo3}
\end{minipage}
\hfill
\begin{minipage}[t]{.45\textwidth}
  \centering
        \includegraphics[width=1\textwidth]{histo4}
		\caption{\footnotesize NLOS, Polarización horizontal, con gente.}
\label{fig:histo4}
\end{minipage}
\end{figure}
\footnotesize
Para este caso, ambas figuras tienen sus máximos en torno a los -25 dB, pero para el caso
con gente transitando la señal se ve más dispersa por los elementos que provocan los 
desvanecimientos.\\

\normalsize
Comparación en escenario LOS en altura en figura \ref{fig:histo26} y \ref{fig:histo27}.
\begin{figure}[H]
\hfill
\begin{minipage}[t]{.45\textwidth}
  \centering
        \includegraphics[width=1\textwidth]{histo26}
		\caption{\footnotesize LOS, Polarizacion horizontal, con gente.}
\label{fig:histo26}
\end{minipage}
\hfill
\begin{minipage}[t]{.45\textwidth}
  \centering
        \includegraphics[width=1\textwidth]{histo27}
		\caption{\footnotesize LOS, Polarización vertical, sin gente.}
\label{fig:histo27}
\end{minipage}
\end{figure}
\footnotesize
Acá aparece una situación un poco extraña, en la figura \ref{fig:histo26} los pick de señal 
estan a una cantidad de dB menores que los de la figura \ref{fig:histo27} lo cual es esperable,
la situación es que en el segundo caso, aparece un máximo total, y un máximo local, donde esto no es
esperable, pero se puede argumentar debido a que la hora donde se hizo la medición ya no era
una situación casi perfecta, transitaba bastante gente (pero menor en consideración a los cambios
de bloque de asignaturas por ejemplo), provocando más multitrayectos, debilitanto la señal.\\

\normalsize
Los siguientes histogramas (figuras \ref{fig:histo7} a figura \ref{fig:histo22}) son del 
procedimiento que se realizó modificando la ubicación de la antena de recepción más o 
menos la distancia de una longitud de onda $\lambda$, como se indicó en el diagrama en la 
tabla numero 2.

\begin{figure}[H]
\hfill
\begin{minipage}[t]{.45\textwidth}
  \centering
        \includegraphics[width=1\textwidth]{histo7}
		\caption{\footnotesize Medición 7.}
\label{fig:histo7}
\end{minipage}
\hfill
\begin{minipage}[t]{.45\textwidth}
  \centering
        \includegraphics[width=1\textwidth]{histo8}
		\caption{\footnotesize Medición 8.}
\label{fig:histo8}
\end{minipage}
\end{figure}

\begin{figure}[H]
\hfill
\begin{minipage}[t]{.45\textwidth}
  \centering
        \includegraphics[width=1\textwidth]{histo9}
		\caption{\footnotesize Medición 9.}
\label{fig:histo9}
\end{minipage}
\hfill
\begin{minipage}[t]{.45\textwidth}
  \centering
        \includegraphics[width=1\textwidth]{histo10}
		\caption{\footnotesize Medición 10.}
\label{fig:histo10}
\end{minipage}
\end{figure}

\begin{figure}[H]
\hfill
\begin{minipage}[t]{.45\textwidth}
  \centering
        \includegraphics[width=1\textwidth]{histo11}
		\caption{\footnotesize Medición 11.}
\label{fig:histo11}
\end{minipage}
\hfill
\begin{minipage}[t]{.45\textwidth}
  \centering
        \includegraphics[width=1\textwidth]{histo12}
		\caption{\footnotesize Medición 12.}
\label{fig:histo12}
\end{minipage}
\end{figure}

\begin{figure}[H]
\hfill
\begin{minipage}[t]{.45\textwidth}
  \centering
        \includegraphics[width=1\textwidth]{histo13}
		\caption{\footnotesize Medición 13.}
\label{fig:histo13}
\end{minipage}
\hfill
\begin{minipage}[t]{.45\textwidth}
  \centering
        \includegraphics[width=1\textwidth]{histo14}
		\caption{\footnotesize Medición 14.}
\label{fig:histo14}
\end{minipage}
\end{figure}

\begin{figure}[H]
\hfill
\begin{minipage}[t]{.45\textwidth}
  \centering
        \includegraphics[width=1\textwidth]{histo15}
		\caption{\footnotesize Medición 15.}
\label{fig:histo15}
\end{minipage}
\hfill
\begin{minipage}[t]{.45\textwidth}
  \centering
        \includegraphics[width=1\textwidth]{histo16}
		\caption{\footnotesize Medición 16.}
\label{fig:histo16}
\end{minipage}
\end{figure}

\begin{figure}[H]
\hfill
\begin{minipage}[t]{.45\textwidth}
  \centering
        \includegraphics[width=1\textwidth]{histo17}
		\caption{\footnotesize Medición 17.}
\label{fig:histo17}
\end{minipage}
\hfill
\begin{minipage}[t]{.45\textwidth}
  \centering
        \includegraphics[width=1\textwidth]{histo18}
		\caption{\footnotesize Medición 18.}
\label{fig:histo18}
\end{minipage}
\end{figure}

\begin{figure}[H]
\hfill
\begin{minipage}[t]{.45\textwidth}
  \centering
        \includegraphics[width=1\textwidth]{histo19}
		\caption{\footnotesize Medición 19.}
\label{fig:histo19}
\end{minipage}
\hfill
\begin{minipage}[t]{.45\textwidth}
  \centering
        \includegraphics[width=1\textwidth]{histo20}
		\caption{\footnotesize Medición 20.}
\label{fig:histo20}
\end{minipage}
\end{figure}

\begin{figure}[H]
\hfill
\begin{minipage}[t]{.45\textwidth}
  \centering
        \includegraphics[width=1\textwidth]{histo21}
		\caption{\footnotesize Medición 21.}
\label{fig:histo21}
\end{minipage}
\hfill
\begin{minipage}[t]{.45\textwidth}
  \centering
        \includegraphics[width=1\textwidth]{histo22}
		\caption{\footnotesize Medición 22.}
\label{fig:histo22}
\end{minipage}
\end{figure}

Respecto a los gráficos anteriores se puede decir que la señal percibida no debiera variar tanto
bajo una situación controlada, lo que sucede en este caso es que dependiendo de como llegue la 
onda al receptor (cuantas longitudes de onda varía desde el transmisor al receptor), es como va 
a variar la calidad de la señal, además cabe considerar que en el momento de toma de mediciones 
el escenario no era tan controlado, debido principalmente a factores externos como gente transitando.
Dicho lo anterior se comenta que por ejemplo para los casos en las figuras \ref{fig:histo11} 
y \ref{fig:histo12} la señal se encuentra levemente debilitada pero altamente concentrada en un 
solo punto, a diferencia por ejemplo de lo que se observa en la figura \ref{fig:histo17} donde acá 
la recepción de la señal es bastante difusa. También, por ejemplo para el caso de la figura 
\ref{fig:histo22} la señal se observa que percibe dos potencias diferenciables, una centrada a los 0
dB por el LOS, y otra en torno a los -10 dB, que deben ser la recepción por los multitrayectos.


\paragraph{5.- Construya gráficos de puntos dispersos de K vs. ambos tiempos de coherencia para
todos los intervalos de 5 minutos medidos.\\}

Para graficar los puntos dispersos de K vs el tiempo de coherencia se capturaron los datos 
de ambas funciones en 2 matrices K y TC y luego se realizo un plot de puntos de cada 
columna usando el comando:
\begin{verbatim}
plot(K(:,i),TC(:,i),’o’);
\end{verbatim}

donde i =2,3,4 representando cada segmento de las mediciones.\\

A continuación en las figuras \ref{fig:p5si1} a \ref{fig:p5si3} se muestran los gráficos 
obtenidos.\\

\begin{figure}[H]
  \centering
        \includegraphics[width=0.65\textwidth]{p5si1}
		\caption{\footnotesize
		Gráfico K vs. Tiempo de coherencia tramo 1 (0-5 [min]).}
\label{fig:p5si1}
\end{figure}

\begin{figure}[H]
  \centering
        \includegraphics[width=0.65\textwidth]{p5si2}
		\caption{\footnotesize
		Gráfico K vs. Tiempo de coherencia tramo 1 (5-10 [min]).}
\label{fig:p5si2}
\end{figure}

\begin{figure}[H]
  \centering
        \includegraphics[width=0.65\textwidth]{p5si3}
		\caption{\footnotesize
		Gráfico K vs. Tiempo de coherencia tramo 1 (10-15 [min]).}
\label{fig:p5si3}
\end{figure}

Se observa una dispersión absoluta de los puntos en cada gráfico, por lo que se infiere 
que el valor de K es bastante inestable con respecto al tiempo de coherencia.\\


\paragraph{6.- Si trabajó con el equipo de 2.4GHz determine el aumento de potencia recibida al 
reemplazar en el receptor la antena dipolo por la directiva. Este aumento debería ser igual a 
la diferencia de ganancias de las antenas en un ambiente de espacio libre, pero en un ambiente 
con presencia significativa de multitrayectoria (o sea específicamente los casos sin línea de 
vista), la diferencia puede ser menor o mayor, según las direcciones de arribo de los diferentes 
rebotes.\\}
No se realiza reemplazo de dipolo, por lo que este punto no se considera.


\newpage
\section{Conclusiones}
\begin{spacing}{1.5}
En los escenarios NLOS la señal alcanza al receptor por medio del multitrayecto de la señal 
es decir las reflexiones y dispersiones, con la correspondiente pérdida de potencia.

Consecuencia de ello debiere haber una mayor autocovarianza que en línea vista y esto se refleja 
en algunos escenarios donde no hay gente y se tiene un factor K grande.  A su vez se destaca 
que la polarización horizontal posee una mayor inestabilidad que la vertical, tal como lo muestran 
las 2 primeras mediciones (LOS sin gente).\\

Cabe señalar, que todo lo anterior se reflejó solo en algunas mediciones. Considerando
que constantemente transitaba gente cerca de la antena receptora, independiente si había tráfico 
alto o bajo de personas, lo cual afectaba la recepción de la señal mostrando algunos resultados 
con un factor K pequeño y gráficos de autocovarianza inestables. De igual modo algunas 
mediciones se realizaron en solo 10 minutos.\\

Finalmente se puede decir que las mediciones realizadas son un tanto imprecisas como se observó en 
la estadística realizada, además, agregar que en comunicaciones inalámbricas son más 
complejas de lo que parece, con una gran cantidad de variables presentes, multitrayectos, 
desvanecimientos temporales y/o espaciales, entre otros dentro del entorno en que se
esté, no es tan simple como un usuario común observa que su notebook se conecta al wifi que 
entrega un access point.

\end{spacing}

\newpage
\section{Referencias}

- All Data Sheets Web, http://www.alldatasheet.com \\

- Wikipedia, http://es.wikipedia.org/wiki/LabVIEW \\
 
- Archivos proporcionados para la experiencia \\

\qquad Mediciones reales a utilizar en el informe previo (mediciones.dat).

\qquad Marco Teórico.

\qquad Antecedentes sobre los equipos.

\qquad Instrumento Virtual Frankonia Power Meter.


\end{document}
